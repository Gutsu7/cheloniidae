\documentclass{article}
\usepackage{spencer}

\noindentpars

\title{Cheloniidae Turtle Graphics}
\author{Spencer Tipping}

\begin{document}
  \maketitle

  \tableofcontents

  \section{Introduction}
    \label{sec:introduction}

    Cheloniidae is a high-performance turtle graphics platform that is focused on generating antialiased, complex images. To this end, it has support for
    multiple turtles, turtle networks, 3D coordinate spaces, translucent lines, and optically-accurate projective rendering.

    A Cheloniidae scene is built by creating a drawing window and then adding turtles and other drawing objects to the window. Turtles are then driven by using
    a series of monadic invocations. This has the advantage that driving one turtle and driving multiple turtles can be achieved using the same syntax. (See
    section \ref{sec:turtle-monad})

    \begin{scalacode}
package cheloniidae

import java.awt.Color
import java.awt.Frame
    \end{scalacode}

  \section{Geometric Logic}
    \label{sec:geometric-logic}

    Before implementing a turtle it is necessary to define a library of geometric primitives. Vectors are the most prominent notion here. Nothing too special is
    happening; it's just a vector of three doubles.

    The other notion is that of a line segment. A basic line segment just consists of two vectors that delineate its endpoints, though later on we'll subclass
    it to add colors.

    \subsection{Vectors}
      \label{sec:vectors}

      Vectors are immutable. The only other unusual thing about them is the presence of the {\tt +*} operator, which stands for ``add-scaled.'' Since it takes
      two parameters, it cannot be used in an infix manner; thus, for instance, its invocation would look like this: {\tt x.+*(y, 2)}.

      \begin{scalacode}
case class Vector (x: Double, y: Double, z: Double) {
  def map (f: (Double, Double) => Double) (v: Vector) = v match {
    case Vector (vx, vy, vz) => Vector (f (x, vx), f (y, vy), f (z, vz))}

  def + (that: Vector) = this.map ((a: Double, b: Double) => a + b) (that)
  def - (that: Vector) = this.map ((a: Double, b: Double) => a - b) (that)
  def * (that: Vector) = this.map ((a: Double, b: Double) => a * b) (that)
  def / (that: Vector) = this.map ((a: Double, b: Double) => a / b) (that)

  def + (a: Double) = new Vector (x + a, y + a, z + a)
  def - (a: Double) = new Vector (x - a, y - a, z - a)
  def * (a: Double) = new Vector (x * a, y * a, z * a)
  def / (a: Double) = new Vector (x / a, y / a, z / a)

  def +* (that: Vector, x: Double) = this.map ((a: Double, b: Double) => a + b * x) (that)

  def dot (that: Vector) = that match {
    case Vector (vx, vy, vz) => x*vx + y*vy + z*vz}

  def cross (that: Vector) = that match {
    case Vector (vx, vy, vz) => Vector (y*vz - z*vy, z*vx - x*vz, x*vy - y*vx)}

  def proj (that: Vector) = that * (this dot that) / (that dot that)
  def orth (that: Vector) = this - this proj that

  def length = Math.sqrt (this dot this)

  def to (that: Vector, c: Color) = new LineSegment (this, that, c)

  override def toString = "<" + x.toString + ", " + y.toString + ", " + z.toString + ">"
}
      \end{scalacode}

    \subsection{Line segments}
      \label{sec:line-segments}

      Line segments, like vectors, are immutable. They provide some handy methods such as directional scaling, length and midpoint calculation, etc. One method
      of note is the V method, which concatenates two lines. This can be used to close a polygon, for instance. It has the property that two lines $A = (v_1,
      v_2)$ and $B = (v_3, v_4)$ combine to $A~V~B = (v_1, v_4)$.

      \begin{scalacode}
case class LineSegment (v1: Vector, v2: Vector, c: Color) {
  def midpoint = (v1 + v2) / 2.0
  def length   = (v1 - v2).length

  def <*> (x: Double) = new LineSegment (midpoint + (v1 - midpoint) * x, midpoint + (v2 - midpoint) * x, c)
  def  *> (x: Double) = new LineSegment (v1, midpoint + (v2 - midpoint) * x, c)
  def <*  (x: Double) = new LineSegment (midpoint + (v1 - midpoint) * x, v2, c)

  def * (x: Double) = v1 * (1.0 - x) + v2 * x

  def V (l: LineSegment) = l match {
    case LineSegment (lv1, lv2, _) => new LineSegment (v1, lv2, c)}
}
      \end{scalacode}

  \section{Turtles}
    \label{sec:turtles}

    Turtles are abstracted at several levels. At the very top, they are things that have the ability to modify a list of line segments that is tracked by the
    containing window. This is useful so that the window can keep a record of all line segments added and perform intermittent redraws as the scene is
    developed.

    The next level abstraction is that of a monad -- this is to say that turtles are immutable data structures that know how to return instances of themselves
    after certain transformations have taken place. Further, not just turtles work this way. A group of connected turtles can be given a bunch of commands
    simultaneously.

    \subsection{Turtle taxonomy}
      \label{sec:turtle-taxonomy}

      In the class hierarchy there are several different interfaces that can be implemented. First, there is the {\tt Turtle} interface defined in section
      \ref{sec:turtle-monad}. This allows different types of turtles to be defined; for example, turtles that use polar angles instead of Cartesian directions,
      or probabilistic turtles whose drawing operations are inexact.

      Turtles have turtle-specific mutator functions, such as {\tt fd} or {\tt rt}, but they can also accept other commands that the user defines. These
      secondary commands

      The normal turtle is the {\tt DirectionalTurtle}, which implements what most people would expect from a turtle.

    \subsection{Turtle Monad}
      \label{sec:turtle-monad}

      \begin{scalacode}
trait Turtle {
  def << (f: Turtle => Turtle) = f (this)
  def >> (f: LineSegment => Unit)
}
      \end{scalacode}

    \subsection{Basic turtle implementation}
      \label{sec:basic-turtle-implementation}

      A geometric turtle is represented as an object whose data structures are a positional vector, represented either by $p$ or by $\langle x, y, z \rangle$,
      and a directional vector, represented either by $d$ or by $\langle dx, dy, dz \rangle$. Operations performed on the turtle yield a new turtle and are
      expected to have side-effects on the turtle's drawing window. In addition to its geometric attributes, a turtle also has a color and a drawing window.


      The trait defined in section \ref{sec:turtle-monad} provides the standard interface for turtle-like things, but we need a solid implementation to store
      the data. Well, here is such an implementation:

      \begin{scalacode}
class GeometricTurtle (p: Vector, d: Vector, c: Color) extends Turtle {
  
}
      \end{scalacode}

    \subsection{Turtle networks}
      \label{sec:turtle-networks}

      Multiple turtles can be coordinated and can draw lines between each other. This is useful when constructing grids, for instance. There are several types
      of turtle networks, and they connect turtles in different ways.

      \begin{scalacode}
class TurtleNetwork extends Turtle {
  var turtles: List[Turtle] = Nil
  def << (l: List[Turtle]) = {this <<< l}
  def << (t: Turtle) = {ts = t :: ts; this}
  protected abstract def <<< (l: List[Turtle])
  protected abstract def <<< (f: Turtle => Turtle)
  protected abstract def <<< (f: Synchronize)
}
      \end{scalacode}

      By default, a network will distribute commands across all turtles. Thus we have a completely distributive network:

      \begin{scalacode}
class CompletelyDistributiveNetwork extends TurtleNetwork {
  protected abstract def <<< (f: Turtle => Turtle) = this << turtles.map (f)
}
      \end{scalacode}

      \subsubsection{Linear turtle network}
        \label{sec:linear-turtle-network}

        A linear turtle network is used when you have $n$ turtles and you wish to form an $n\times m$ grid. The idea is that at any point you can invoke the
        {\tt synchronize} operation on the network and each turtle will draw a line to its neighbor. Thus a standard use case might look something like this:

        \begin{verbatim}
val w = new TurtleDrawingWindow ()
val n = new LinearNetwork ()
w << n
val t1 = w.createTurtle ()
val t2 = t1 + Vector (50, 0, 0)
n << fd(100) << synchronize << rt(10) << fd(100) << synchronize
        \end{verbatim}

        \begin{scalacode}
class LinearNetwork extends CompletelyDistributiveTurtleNetwork {
  def <<< (f: Synchronize) = 
}
        \end{scalacode}

    \subsection{Drawing primitives}
      \label{sec:drawing-primitives}

      A turtle wouldn't be of much use without some primitives to work with. These drawing primitives allow the turtle to render lines and change its direction.

      \begin{scalacode}
object Drawing {
  def changePosition (pprime: Vector) = (t: Turtle) => t match {
    case GeometricTurtle (p, d, c, l) => {w.addLineSegment (p, pprime, c)
                                          new GeometricTurtle (pprime, d, c, l)}}

  def changeDirection (dprime: Vector) = (t: Turtle) => t match {
    case GeometricTurtle (p, d, c, l) => new GeometricTurtle (p, dprime, c, l)}

  def fd (x: Double) = (t: Turtle) => t match {
    case GeometricTurtle (p, d, c, l) => changePosition (p + d * x) (t)}
}
      \end{scalacode}

  \section{Turtle Drawing Window}
    \label{sec:turtle-drawing-window}

    A turtle drawing window maintains a list of line segments generated by turtles and displays them onscreen. It also can generate a turtle with sensible
    defaults.

    \begin{scalacode}
class TurtleDrawingWindow extends Frame with LineReceiver {
}
    \end{scalacode}

% vim: set syntax=scalatex :
\end{document}
