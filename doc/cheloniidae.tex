\documentclass{article}

\usepackage{palatino}
\usepackage{amsmath}
\usepackage{mathpazo}
\usepackage{listings}
\usepackage{hyperref}

\lstnewenvironment{java}
  {\lstset{language=java,basicstyle={\tt\scriptsize}}}
  {}

\newenvironment{commandtable}
  {\begin{enumerate}}
  {\end{enumerate}}

\def\command[#1]{\item[\tt #1]}
\def\example[#1]{\par Example: {\tt #1}}

\begin{document}
  \title{Cheloniidae Turtle Graphics}
  \author{Spencer Tipping}
  \date{}
  \maketitle

  \tableofcontents

  \section{Introduction}
    \label{sec:introduction}

    Cheloniidae is a three-dimensional turtle graphics library for Java. It provides turtle replication, distributed commands, stable rendering, and an
    extensible framework for using alternate coordinate models. In addition, it provides basic control flow constructs for iteration and recursion.

    \subsection{Mouse and Keyboard Commands}
      \label{sec:mouse-and-keyboard-commands}

      These are the basic interactions that the turtle window supports:

      \begin{enumerate}
        \item[\bf Mouse drag]         Moves the camera in 3D space.
        \item[\bf Shift + mouse drag] Rotates the camera about the center of the structure being drawn.
        \item[\bf Mouse wheel]        Zooms the camera forwards or backwards. (After a zoom, a click is required to redraw the structure fully.)
        \item[\bf Control]            Reduces the speed of any other action.
      \end{enumerate}

      When dragging, the scene is redrawn using a point cloud representing only a subset of the total number of lines. This is to ensure that the interface
      continues to respond quickly even for large scenes. To change the number of lines drawn, see the {\tt intermediatePointCloudSize} attribute of {\tt
      cheloniidae.TurtleWindow}.

  \section{Turtle Commands}
    \label{sec:turtle-commands}

    Starting with Cheloniidae 3, there is a new interface for using turtles. Section \ref{sec:using-the-standard-rotational-turtle} gives a quick overview of
    the new interface, and the following sections describe the underlying design decisions that led to it.

    \subsection{Using the Standard Rotational Turtle}
      \label{sec:using-the-standard-rotational-turtle}

      Instead of driving turtles directly, turtle commands let you construct a series of actions for a turtle to perform. Cheloniidae will send the actions to
      the turtle for you, which means that you don't have to create a turtle or a turtle window. Here is the basic setup code:

      \begin{java}
import cheloniidae.*;
import cheloniidae.frames.*;
public class MyClass {
  public static void main (String[] args) {new MyClass ();}
  public TurtleCommand commands () {
    // Return a TurtleCommand object here.
  }
}     \end{java}

      \subsubsection{Basic Commands}
        \label{sec:basic-commands}

        Here are the simple commands understood by the standard rotational turtle:

        \begin{enumerate}
          \command[move(x)]  Moves the turtle $x$ units in its {\tt direction}, creating a line.
          \command[jump(x)]  Moves the turtle $x$ units in its {\tt direction} without creating a line.
          \command[turn(x)]  Turns the turtle sideways by $x$ degrees.
          \command[bank(x)]  Tilts the turtle around its {\tt direction} by $x$ degrees. Note that this doesn't actually change the turtle's direction, but it
                             does affect future {\tt turn} and {\tt pitch} commands.
          \command[pitch(x)] Pitches the turtle (vertical rotation) by $x$ degrees.
          \command[pass()]   Does nothing.
        \end{enumerate}

        In addition to actions, some basic combinations are provided as well. These commands can be used anywhere a normal turtle command is expected.

        \begin{enumerate}
          \command[sequence(c1, c2, c3, ...)] Performs each of its commands in sequence.
          \command[repeat(n, c1, c2, ...)]    Repeats the sequence of commands $(c1, c2, \cdots)$ $n$ times.
        \end{enumerate}

    \subsection{History}
      \label{sec:history}

      In Cheloniidae version 2.x, turtles were driven by direct commands, like this:

      \begin{java}
import cheloniidae.*;
public class square {
  public static void main (String[] args) {
    StandardRotationalTurtle t = new StandardRotationalTurtle ();
    TurtleWindow             w = new TurtleWindow ();
    w.add (t);
    for (int i = 0; i < 4; ++i) t.move (50).turn (90);
    w.setVisible (true);
  }
}     \end{java}

      Version 3 provides a more intuitive and powerful interface. {\tt square} should now be written this way:

      \begin{java}
import cheloniidae.*;
import cheloniidae.frames.*;
public class square extends SingleTurtleScene {
  public static void main (String[] args) {new square ();}
  public TurtleCommand commands () {
    return repeat (4, move (50), turn (90));
  }
}     \end{java}

      This has two advantages. First, it does all of the window and turtle setup for you; and second, any command that is given to one turtle can be given to
      multiple turtles simultaneously. (See the {\tt replicatedtube} example file, for instance.)

    \subsection{Turtles and Geometry}
      \label{sec:turtles-and-geometry}

      \begin{quote}
      {\em Note: This section is primarily of academic interest. To begin using Cheloniidae quickly, I recommend skipping to section
      \ref{sec:using-the-standard-rotational-turtle}.}
      \end{quote}

      Cheloniidae makes few assumptions about geometric axioms. While currently only Euclidean geometry is used, it would not be difficult to add turtles that
      exhibited non-Euclidean behavior. Thus each aspect of turtle behavior was built on a minimalistic foundation to allow maximum flexibility.

      The lowest-level turtle included with Cheloniidae is the Euclidean turtle, which has a position and direction each specified by a 3D vector. Specification
      of the direction is left to the subclass, since there are several different models that could be used. The only direct subclass provided is the Cartesian
      turtle, which stores its direction as a vector instance.

\end{document}
